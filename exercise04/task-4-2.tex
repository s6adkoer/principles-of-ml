\documentclass[10pt,a4paper]{article}
\usepackage[left=0.75cm, right=0.75cm]{geometry}
\usepackage[utf8]{inputenc}
\usepackage[T1]{fontenc}
\usepackage{lmodern}
\usepackage{colortbl}
\usepackage{amsmath,amssymb,amsthm,mathtools}
\usepackage{parskip}
\usepackage{fancyhdr}
\usepackage{chessboard}
\usepackage{hyperref}
\usepackage{mathtools}
\usepackage{mathrsfs}
\usepackage{tikz}
\usepackage{tikz-cd}
\usetikzlibrary{decorations.pathmorphing}
\usetikzlibrary{calc}
\usetikzlibrary{babel}
\usetikzlibrary{cd}
\usetikzlibrary{positioning}
\usetikzlibrary{calc}
\newtheorem{lemma}{Lemma}
\newtheorem{corollary}[lemma]{Corollary}
\linespread{1,25}
\pagestyle{fancy}
\renewcommand{\headrulewidth}{0pt}
\definecolor{grey}{rgb}{0.8,0.8,0.8}
\newcommand {\N} {\mathbb{N}}
\newcommand {\Z} {\mathbb{Z}}
\newcommand {\Q} {\mathbb{Q}}
\newcommand {\R} {\mathbb{R}}
\newcommand {\C} {\mathbb{C}}
\newcommand {\F} {\mathbb{F}}
\newcommand {\iN} {\in \mathbb{N}}
\newcommand {\iZ} {\in \mathbb{Z}}
\newcommand {\iQ} {\in \mathbb{Q}}
\newcommand {\iR} {\in \mathbb{R}}
\newcommand {\iC} {\in \mathbb{C}}
\newcommand {\iF} {\in \mathbb{F}}
\newcommand {\Ra} {\implies}
\newcommand{\powerset}{\mathcal{P}}
\newcommand{\A}{\mathcal{A}}
\newcommand{\B}{\mathcal{B}}
\newcommand{\tr}{\text{tr}}
\newcommand*{\dotbigcup}{\mathop{\dot\bigcup}}
\newcommand{\dotcup}{\mathbin{\dot\cup}}
\DeclareMathOperator{\e}{e}
\DeclareMathOperator{\osz}{osz}
\let\i\undefined
\DeclareMathOperator{\i}{i}
\DeclareMathOperator{\Var}{Var}
\renewcommand{\d}{\;\mathrm{d}}
\DeclareMathOperator{\Vol}{Vol}
\DeclareMathOperator{\End}{End}
\DeclareMathOperator{\id}{id}
\DeclareMathOperator{\cyl}{cyl}
\DeclareMathOperator{\Bild}{Bild}
\DeclareMathOperator{\pr}{pr}
\DeclareMathOperator{\ord}{ord}
\usepackage{mathrsfs}
\DeclareMathAlphabet{\mymathbb}{U}{bbold}{m}{n}
\newcommand{\1}{\mymathbb{1}}
\newcommand{\M}{\mathcal{M}(\mu^\ast)}
\newcommand{\Nm}{\mathcal{N}(\mu)}
\DeclarePairedDelimiter{\norm}{\|}{\|}
\newcommand {\func}[5] {\left(\begin{array}{crcl}#1:&#2&\to&#3\\&#4&\mapsto&#5\end{array}\right)}
\renewcommand{\comment}[1]{}

\pagestyle{empty}

\begin{document}


\section*{Task 4.2}

\begin{lemma} \label{l1}
    For a matrix $A\in\R^{n\times n}$, $w=\frac{1}{n}1_n\in \R^n$ and $z\in\R^n$ the following equality holds:
    \[\tr(Awz^\intercal)=z^\intercal A w\]
\end{lemma}

\begin{proof}
    We can ignore the $\frac{1}{n}$, because it commutes with all of our operations. Therefore we show:
    \[\tr(A1_nz^\intercal)=z^\intercal A 1_n\]
    using the well know fact, that the trace operator is invariant under cyclic permutations, which, for two matrices $A,B\in\R^{n\times n}$, implies 
    \[\tr(AB)=\tr(BA).\]
    Therefore
    \begin{align*}
        \tr(A1z^\intercal)&=\tr(1z^\intercal A)\\
        &\stackrel{\star}{=}\sum_{i=1}^n(z^\intercal A)_i
        \stackrel{\star\star}{=}\underbrace{(z^\intercal A)1_n}_{\text{scalar product}}
    \end{align*}

    $\star$ holds, because (by associativity) $1_nz^\intercal A=1_n\underbrace{(z^\intercal A)}_{\eqqcolon a\text{, row vector}}$, i.e. 
    $1_nz^\intercal A$ can be expressed as a product of the column vector $1$ and $a$, which is just a $n\times n$ matrix $M$,
    where each of the $n$ rows is given by $a$. Therefore the trace (sum of the diagonal elements) is just the sum of the elements of $a$:
    \[\tr(M)=\sum_{i=1}^n M_{ii}=\sum_{i=1}^n a_i\]

    $\star\star$ holds by the definition of the scalar product:

    \[\sum_{i=1}^n(z^\intercal A)_i=\sum_{i=1}^n(z^\intercal A)_i\cdot 1 =\sum_{i=1}^n(z^\intercal A)_i(1_n)_i=(z^\intercal A)1_n \]

\end{proof}

\begin{corollary}
    For a data matrix $X$, $w=\frac{1}{n}1_n\in \R^n$ and $z\in\R^n$ the following equality holds:
    \[\tr(X^\intercal Xwz^\intercal)=z^\intercal X^\intercal X w\]
\end{corollary}

\begin{proof}
    The claim is the result of lemma \ref{l1} for $A=X^\intercal X$.
\end{proof}

\begin{corollary}
    For a data matrix $X$, $w=\frac{1}{n}1\in \R^n$ and $z\in\R^n$ the following equality holds:
    \[\tr(zw^\intercal X^\intercal Xwz^\intercal)=z^\intercal zw^\intercal X^\intercal Xwz^\intercal\]
\end{corollary}

\begin{proof}
    The claim is the result of lemma \ref{l1} for $A=zw^\intercal X^\intercal X$, since 
    $zw^\intercal\in \R^{n\times n}$ and therefore $zw^\intercal \underbrace{ X^\intercal X}_{\in \R^{n\times n}} \in \R^{n\times n}$
\end{proof}

\end{document}
